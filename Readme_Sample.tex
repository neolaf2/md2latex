\documentclass{article}
\usepackage[utf8]{inputenc}
\usepackage[T1]{fontenc}
\usepackage{lmodern}
\usepackage{hyperref}
\usepackage{listings}
\usepackage{xcolor}

\title{md2latex VSCode Extension}
\date{}

\begin{document}
\maketitle

\section{Overview}
The \texttt{md2latex} extension enables you to easily convert between Markdown and LaTeX documents directly within Visual Studio Code using Pandoc as the conversion engine.

\section{Features}
\begin{itemize}
  \item \textbf{Markdown to LaTeX:} Convert \texttt{.md} files to \texttt{.tex} files.
  \item \textbf{LaTeX to Markdown:} Convert \texttt{.tex} files to \texttt{.md} files.
  \item \textbf{Context Menu Integration:} Right-click on a file in the Explorer or editor tab to quickly trigger a conversion.
  \item \textbf{Command Palette Support:} Use VS Code's Command Palette (\texttt{Ctrl+Shift+P} on Windows/Linux or \texttt{Cmd+Shift+P} on macOS) to run conversion commands.
\end{itemize}

\section{Prerequisites}
\begin{itemize}
  \item \textbf{Pandoc:} Ensure Pandoc is installed and available in your system's PATH. You can verify the installation by running:
  \begin{lstlisting}[basicstyle=\ttfamily,breaklines=true]
pandoc --version
  \end{lstlisting}
  Download Pandoc from \href{https://pandoc.org/installing.html}{pandoc.org}.
  
  \item \textbf{Visual Studio Code:} Make sure you have VS Code installed.
\end{itemize}

\section{Installation}
\subsection*{Option 1: Install from Source}
\begin{enumerate}
  \item \textbf{Clone the Repository:}
  \begin{lstlisting}[basicstyle=\ttfamily,breaklines=true]
git clone https://github.com/neolaf2/md2latex.git
cd md2latex
  \end{lstlisting}
  
  \item \textbf{Install Dependencies:}
  \begin{lstlisting}[basicstyle=\ttfamily,breaklines=true]
npm install
  \end{lstlisting}
  
  \item \textbf{Build the Extension (if applicable):}\\
  If using TypeScript, compile the project:
  \begin{lstlisting}[basicstyle=\ttfamily,breaklines=true]
npm run compile
  \end{lstlisting}
  
  \item \textbf{Launch the Extension:}\\
  Open the project in VS Code and press \texttt{F5} to open a new Extension Development Host window with the extension loaded.
\end{enumerate}

\subsection*{Option 2: Install from the VS Code Marketplace}
If the extension is published, search for \texttt{md2latex} in the Extensions view (\texttt{Ctrl+Shift+X} on Windows/Linux or \texttt{Cmd+Shift+X} on macOS) and click \textbf{Install}.

\section{Usage}
\begin{enumerate}
  \item \textbf{Open a Document:}\\
  Open a Markdown (\texttt{.md}) or LaTeX (\texttt{.tex}) file in VS Code.
  
  \item \textbf{Trigger Conversion:}
  \begin{itemize}
    \item \textbf{Context Menu:} Right-click the file in the Explorer or editor title area.
      \begin{itemize}
        \item For a Markdown file, select \texttt{Convert Markdown to LaTeX}.
        \item For a LaTeX file, select \texttt{Convert LaTeX to Markdown}.
      \end{itemize}
    \item \textbf{Command Palette:} Press \texttt{Ctrl+Shift+P} (or \texttt{Cmd+Shift+P} on macOS) and search for the conversion command.
  \end{itemize}
  
  \item \textbf{View Output:}\\
  The extension uses Pandoc to perform the conversion, creates the new file in the same directory (with the appropriate file extension), and automatically opens the converted document in a new editor tab.
\end{enumerate}

\section{Contributing}
Contributions are welcome! Feel free to fork the repository, submit issues, or create pull requests. For any questions or suggestions, please open an issue on the GitHub repository:
\begin{center}
\url{https://github.com/neolaf2/md2latex}
\end{center}

\section{License}
This project is licensed under the MIT License.

\end{document}